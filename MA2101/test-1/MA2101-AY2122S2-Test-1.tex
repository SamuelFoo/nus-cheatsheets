\documentclass[a4paper]{article}
\usepackage[a4paper,
            top=0.4in,
            bottom=0.6in,
            left=0.4in,
            right=0.4in,
            landscape]{geometry}

% default stuff
\usepackage{amsmath}
\usepackage{amssymb}
\usepackage{enumitem}

% multicolumn package
\usepackage{multicol}

\usepackage{multirow}
\usepackage{makecell}
\renewcommand\theadfont{\bfseries}

\newcolumntype{M}[1]{>{\centering\arraybackslash} m{#1}} % centered m

\newcommand{\abs}[1]{\left\lvert#1\right\rvert}

\newcommand{\ol}[1]{\begin{enumerate}#1\end{enumerate}}
\newcommand{\oll}[1]{\begin{enumerate}[leftmargin=*]#1\end{enumerate}}
\newcommand{\ul}[1]{\begin{itemize}#1\end{itemize}}
\newcommand{\ull}[1]{\begin{itemize}[leftmargin=*]#1\end{itemize}} % no margin

\usepackage{spalign}

\let\mat=\spalignmat
\let\amat=\spalignaugmat

\newcommand{\detmat}[1]{\spaligndelims\vert\vert\spalignmat{#1}\spaligndelims()}

\newcommand{\ro}[1]{\xrightarrow{\begin{subarray}{l}#1\end{subarray}}}
\newcommand{\roTwo}[2]{\xrightarrow{\begin{subarray}{l}#1\\#2\end{subarray}}}
\newcommand{\roThree}[3]{\xrightarrow{\begin{subarray}{l}#1\\#2\\#3\end{subarray}}}

% matrix operations
\newcommand{\rref}{\mathrm{rref}}

\newcommand{\transpose}{^{\text{T}}}
\newcommand{\tr}{^{\text{T}}}

\newcommand{\GJE}{\ro{\text{GJE}}}
\newcommand{\matdim}[1]{\mathrm{dim}(#1)}
\newcommand{\matrank}[1]{\mathrm{rank}(#1)}
\newcommand{\matnullity}[1]{\mathrm{nullity}(#1)}

\newcommand{\minrank}[1]{\mathrm{min}\{#1\}}

\newcommand{\vabs}[1]{\abs{\abs{#1}}}

\newcommand{\by}[2]{$#1 \times #2$}
\newcommand{\laspan}[1]{\mathrm{span}\left\{#1\right\}}

\newcommand{\spc}{\spaligndelims..} % spalign cancel brackets
\newcommand{\spv}{\spaligndelims\vert\vert} % spalign vert
\newcommand{\spr}{\spaligndelims()} % spalign reset

% vector bold
\newcommand{\vb}[1]{\boldsymbol{#1}}
\newcommand{\I}{\vb{I}}
\newcommand{\vz}{\vb{0}}

% expand to row vector
% #1 is the vector, #2 is the length
% refer to MA2001 Tutorial 5 for sample usage
\newcommand{\exprow}[2]{
  \rule[.4ex]{#2}{0.5pt}
  \hspace{0.2cm}
  #1
  \hspace{0.2cm}
  \rule[.4ex]{#2}{0.5pt}
}

% set of column vectors
\newcommand{\vset}[1]{\left\{#1\right\}}

% LA
\newcommand{\spn}[1]{\mathrm{span}(#1)}
\newcommand{\rank}[1]{\mathrm{rank}(#1)}
\newcommand{\nullity}[1]{\mathrm{nullity}(#1)}
\newcommand{\lm}{\lambda}

% matrices
\newcommand{\A}[1][]{\vb{A_{#1}}}
\newcommand{\B}{\vb{B}}
\newcommand{\E}{\vb{E}}
\newcommand{\D}{\vb{D}}
\renewcommand{\P}{\vb{P}}
\newcommand{\Range}{\mathrm{R}}
\newcommand{\Kernel}{\mathrm{Ker}}
\newcommand{\trc}[1]{\mathrm{tr}(#1)}

% vectors
\renewcommand{\u}[1][]{\vb{u_{#1}}}
\renewcommand{\v}[1][]{\vb{v_{#1}}}
\newcommand{\w}[1][]{\vb{w_{#1}}}
\newcommand{\x}[1][]{\vb{x_{#1}}}
\renewcommand{\b}{\vb{b}}
\newcommand{\n}{\vb{n}}
\newcommand{\p}{\vb{p}}

% coloring code
\usepackage{color}
\definecolor{dkgreen}{rgb}{0,0.6,0}
\definecolor{gray}{rgb}{0.5,0.5,0.5}
\definecolor{mauve}{rgb}{0.58,0,0.82}

\definecolor{lg}{rgb}{0.9,0.9,0.9}

% code environment
\usepackage{listings}
\lstset{
  %frame=tb, % adds top and bottom border
  aboveskip=1mm,
  belowskip=1mm,
  showstringspaces=false,
  columns=flexible,
  basicstyle={\small\ttfamily},
  numberstyle=\color{gray},
  keywordstyle=\color{blue}\textbf,
  commentstyle=\color{dkgreen},
  stringstyle=\color{mauve},
  breaklines=true,
  breakatwhitespace=true,
  tabsize=4
}
\newcommand{\ic}[1]{\lstinline{#1}}

% drawing a stack diagram
\usepackage{drawstack}
\usetikzlibrary{arrows.meta}

\newcommand{\ptr}[1]{${}^{*}$#1}
\newcommand{\addr}[1]{\&#1}
\newcommand{\addrb}[1]{(\&#1)}

% special table column for code
\usepackage{collcell}
\newcommand*{\tablecode}[1]{\lstinline{#1}}% Do anything you like with `#1`
\newcolumntype{T}{>{\collectcell\tablecode}c<{\endcollectcell}}


\renewcommand{\th}{\theta}
\newcommand{\inv}{^{-1}}
\newcommand{\R}{\mathbb{R}}
\newcolumntype{C}{>{$}c<{$}} % math-mode version of "c" column type

\begin{document}
\part*{\centering \underline{MA2101}}
\begin{multicols*}{4}
  \small
  \section*{\underline{Matrices}}
    % 1.2
    \paragraph{Describing change} Let $T$ be a 2D LT.
      \[ \mathcal{M}(T) = \mat{a b; c d} \]
      This means that $T(\hat{i}) = \mat{a;c}$ and $T(\hat{j}) = \mat{b;d}$, i.e. the columns tell us how the basic unit vectors change under the transformation.
    \paragraph{Rotation matrix (2D)} An anticlockwise rotation by $\theta$ is given by
      \[ \mat{\cos\th, -\sin\th; \sin\th, \cos\th} \]
    \paragraph{Shear matrix (2D)} A shear parallel to the $x$-axis by $\theta$ is given by
      \[ \mat{1 \tan\th; 0 1} \]
    % does not include that it can be further decomposed
    \paragraph{Tut1 Q1} A matrix can be \underline{decomposed} as a sum:
      \[ A = \frac{1}{2}(A + A\tr) + \frac{1}{2}(A - A\tr) \]
      The first term is symmetric, and the second term is antisymmetric.
      \\\\
      Also, the second term is traceless in the following:
      \[ B = \frac{\trc{B}}{n} I + \left( B - \frac{\trc{B}}{n} I \right) \]
      Applying this to the first term of the previous equation, we CAN decompose a matrix into: symmetric traceless + multiple of identity + antisymmetric.
    \paragraph{Tut1 Q4} The \underline{exponential} of a matrix is defined as:
      \[ e^A = I + A + \frac{A^2}{2!} + \cdots = \sum^\infty_{i=0} \frac{A^i}{i!} \]
    \paragraph{Tut2 Q1} To check that a vector lies on a plane, the dot product of the vector, and the normal vector, should be 0.
    \paragraph{Tut2 Q5} For a matrix $A$, $\det e^A = e^{\trc{A}}$.
    % 1.4
    \subsection*{Determinants}
      The determinant of a matrix tells us how the area (2D) / volume (3D) of the basic box changes with the transformation associated with the matrix.
      \paragraph{2D} Let $T$ be a 2D LT.
      \[ \abs{\det{T}} = \abs{T\hat{i} \times T\hat{j}} = \abs{T\hat{i}} \abs{T\hat{j}} \sin\theta  \]
        We use this to obtain the formula for the determinant of a \by{2}{2} matrix.
      \paragraph{3D} Let $T$ be a 3D LT.
        \begin{align*}
          \abs{\det{T}} &= \abs{(T\hat{i} \times T\hat{j}) \cdot T\hat{k}} \\
                        &= \abs{T\hat{i} \times T\hat{j}} \abs{T\hat{k}} \cos\theta
        \end{align*}
        This is also known as the triple product.
        \ull {
          \item Same under cyclic perm of vars
          \item Same under swapping $\times$ and $\cdot$
          \item Negates under swapping a pair of vars
        }
      \paragraph{LA1 Defn 2.5.2} The determinant of a \by{n}{n} square matrix $\A$ is defined as:
        \begin{equation*}
          \det(\A) =
          \begin{cases}
            a_{11} & \text{if } n=1 \\
            a_{11}A_{11} + a_{12}A_{12} + \cdots \\ + a_{1n}A_{1n} & \text{if } n>1
          \end{cases}
        \end{equation*}
        where
        \begin{equation*}
          A_{ij} = (-1)^{i+j}\det(\vb{M_{ij}})
        \end{equation*}
        where $\vb{M_{ij}}$ is a \by{(n-1)}{(n-1)} matrix obtained from $\A$ by deleting the $i$th row and $j$th column. The scalar value $A_{ij}$ is called the $(i,j)$-cofactor of $\A$.
      \paragraph{LA1 Defn 2.5.24} The adjoint of a square matrix $\A$ is defined as:
        \begin{equation*}
          \mathrm{adj}(\A) = (A_{ij})\tr_{n \times n}
        \end{equation*}
        where $A_{ij}$ is the $(i,j)$-cofactor of $\A$.
      \paragraph{Properties} Let $A, B$ be square matrices of order $n$, and $c$ a scalar. Then
        \ol {
          \item \( \det{cA} = c^n \det{A} \)
          \item \( \det{AB} = \det{BA} = \det{A} \times \det{B} \)
          \item \( \det{A} = \det{A\tr} \)
          \item If $A$ invertible, then $\det{A\inv} = \dfrac{1}{\det{A}}$.
        }
      % 1.5
      \paragraph{Inverse property}
        Let $A,B,C$ be square matrices of the same order. Then \( (ABC)\inv = C\inv B\inv A\inv \).
  % 1.6 & 1.7 & 1.10
  \section*{\underline{Eigenvectors}}
    Let $T$ be a LT, let $u$ be a vector. If \( Tu = \lm u \) for some scalar $\lm$, then $u$ is an eigenvector of $T$ corresponding to eigenvalue $\lm$.
    \paragraph{Find eigenvalues} Solve $\det(\lm I - A) = 0$.
    \paragraph{Find eigenvectors} Solve $\det(\lm I - A) x = 0$, substituting the specific eigenvalue $\lm$.
    \paragraph{Product of eigenvalues}
      The product of eigenvalues is the determinant. It tells us how much the volume of the basic box changes.
    \paragraph{Sum of eigenvalues} The trace of a matrix, denoted $\trc{A}$, is defined to be the sum of diagonal entries. Note that \( \trc{A} = \sum \lm \).
    \paragraph{Properties of trace} The trace of only makes sense for square matrices. Let $A,B,C,P$ be order $n$ matrices.
      \oll {
        \item Same under cyclic permutations
          \[ \trc{ABC} = \trc{BCA} = \trc{CAB} \]
        \item Same under matrix change of basis
          \[ \trc{P\inv A P} = \trc{A P P\inv} = \trc{A I} = \trc{A} \]
        \item \( \trc{A} = \trc{A\tr} \)
        \item Is a LT, \( T: \mathcal{M}_{n \times n} \rightarrow \R \)
        \item Is surjective (consider \( \frac{k}{n} \trc{I} \))
        \item Is NOT injective (consider change of basis with different $P$)
      }
  % 1.8
  \section*{\underline{Diagonal form of LT}}
    \paragraph{Column vector relative to new basis}
      Consider $\R^2$. Let $\{\hat{i}, \hat{j}\}$ be the standard basis. Let $\{u,v\}$ be another basis. Define
      \begin{align*}
        u &= p^1_1 \hat{i} + p^2_1 \hat{j} = \mat{p^1_1; p^2_1} \\
        v &= p^1_2 \hat{i} + p^2_2 \hat{j} = \mat{p^1_2; p^2_2}
      \end{align*}
      Note that $p^a_b$ means the scalar that belongs to row $a$ and column $b$. Then $P = \mat{p^1_1 p^1_2; p^2_1 p^2_2}$ takes $(\hat{i},\hat{j})$ to $(u,v)$. Since both are bases, then $\det{P} \neq 0$.
      \\\\
      We want to express a vector $x$ using the new basis. We want to find $\alpha,\beta$ such that
      \[ x = \mat{a;b}_{(\hat{i},\hat{j})} = a\hat{i} + b\hat{j} = \alpha u + \beta v = \mat{\alpha; \beta}_{(u,v)}\]
      It is a fact that
      \[ \mat{a;b}_{(\hat{i},\hat{j})} = P \mat{\alpha; \beta}_{(u,v)} \]
      and since $P$ is invertible,
      \[ \mat{\alpha; \beta}_{(u,v)} = P \inv \mat{a;b}_{(\hat{i},\hat{j})} \]
      so we have a way of expressing the vector relative to the new basis.
    \paragraph{Matrix relative to new basis} Let $T$ be a 2D LT, and let $x,y$ be 2D vectors. Declare
      \[ T_{(\hat{i},\hat{j})} x_{(\hat{i},\hat{j})} = y_{(\hat{i},\hat{j})} \]
      and by algebra:
      \begin{align*}
        P\inv T_{(\hat{i},\hat{j})} P P\inv x_{(\hat{i},\hat{j})} &= P\inv y_{(\hat{i},\hat{j})} \\
        \Big( P\inv T_{(\hat{i},\hat{j})} P \Big) P\inv x_{(\hat{i},\hat{j})} &= P\inv y_{(\hat{i},\hat{j})} \\
        \Big( P\inv T_{(\hat{i},\hat{j})} P \Big) x_{(u,v)} &= y_{(u,v)}
      \end{align*}
      so the matrix relative to the new basis is
      \[ \Big( P\inv T_{(\hat{i},\hat{j})} P \Big) \]
    \paragraph{Row vector relative to new basis (Tut2 Q2)} Let $\vb{c}$ be a column vector, and $\vb{r}$ be a row vector. Under a change of basis,
      \[ \vb{c} \rightarrow P\inv\vb{c} \]
      Since $\vb{rc}$ is a number, it has to stay unchanged under a change of basis:
      \[ \vb{rc} \rightarrow (\vb{r}P) (P\inv\vb{c}) = \vb{rc} \]
      So we hypothesize (and it works), that
      \[ \vb{r} \rightarrow \vb{r}P \]
      In a similar fashion, notice $\vb{r}M\vb{c}$ is a number.
      \[ \vb{r}M\vb{c} \rightarrow (\vb{r}P) (P\inv M P) (P\inv\vb{c}) = \vb{r}M\vb{c} \]
      This is another way to make sense of the change of basis formula.
    % 1.9
    \paragraph{Diagonalization} The matrix of a transformation relative to its own eigenvectors (assuming they form a basis) is diagonal, i.e.
      \begin{align*}
        P\inv T P &= D \\
        T &= P D P\inv
      \end{align*}
      and we use this to calculate powers of matrices.
  \section*{\underline{Vector Spaces}}
    % 2.7
    \paragraph{Addition} Addition is a mapping \( f: V \times V \rightarrow V \).
    % 2.9
    \paragraph{Scalar multiplication} Scalar multiplication is a mapping \( \mathcal{F} \times V \rightarrow V \).
    % 2.10
    \paragraph{Axioms} A vector space is a set $V$ with an addition and scalar multiplication such that
      \ull {
        \item Addition is commutative:
          \[ u+v = v+u \quad \forall u,v \in V \]
        \item Addition is associative:
          \[ (u+v)+w = u+(v+w) \quad \forall u,v,w \in V \]
        \item There is an additive identity:
          \[ \exists 0 \in V \quad v+0=v \quad \forall v \in V \]
        \item Every $v \in V$ has an additive inverse:
          \[ \forall v \in V, \; \exists w \in V \quad v+w = 0 \]
        \item There is a multiplicative identity:
          \[ \exists 1 \in \mathcal{F}, \; \forall v \in V \quad 1v = v \]
        \item Multiplication is distributive both ways:
          \[ \forall a,b \in \mathcal{F}, \; \forall u,v \in V \quad a(u+v) = au+av \]
          \[ \forall a,b \in \mathcal{F}, \; \forall u,v \in V \quad (a+b)u = au+bu \]
      }
  \section*{\underline{Subspaces}}
    % 2.11
    \paragraph{Definition} A subset $U$ of a vector space $V$ is a subspace if $U$ is a vector space, with the same scalar multiplication and addition as in $V$.
    \paragraph{Verification} Three things to verify:
      \ull {
        \item Existence of additive identity (zero)
        \item Closed under addition
        \item Closed under scalar multiplication
      }
      The rest of the vector space axioms will follow.
    % 2.12
    \paragraph{Sum of subspaces} Let $U_1, U_2$ be subspaces wrt $V$. Then
      \[ U_1+U_2 = \{ u_1+u_2: u_1 \in U_1, \; u_2 \in U_2\} \]
    % 2.13, 2.14
    \paragraph{Direct sum} If $U_1$ and $U_2$ above were disjoint, then $U_1 + U_2$ is the direct sum of $U_1$ and $U_2$, and is denoted by $U_1 \oplus U_2$.
  \section*{\underline{Isomorphisms}}
    \paragraph{Definitions} Let $F$ be a mapping $F:S \rightarrow T$.
      \ull {
        \item Surjection
          \[ \forall t \in T, \; \exists s \in S \quad F(s) = t \]
        \item Injection
          \[ \forall s_1,s_2 \in S \quad F(s_1) = F(s_2) \Rightarrow s_1 = s_2 \]
        \item Bijection: Surjection and Injection
      }
    \paragraph{Homomorphism} Let $\phi:U \rightarrow V$ be a mapping. It is a homomorphism if
      \begin{align*}
        \phi(u+v) &= \phi(u) + \phi(v) \\
        \phi(au) &= a \phi(u)
      \end{align*}
      If this homomorphism is \underline{also} a bijection, then this is an isomorphism.
    % 2.17
    \paragraph{Infinite isomorphisms} Let $V$ be a vector space. The mapping $v \rightarrow cv \; \forall v \in V$ is an isomorphism, and there are infinitely many different $c$.
    % 2.18
    \paragraph{Finite dimensional} A vector space is finite dimensional over $\mathcal{F}$ if it is isomorphic to $\mathcal{F}^n$ for some finite integer $n$.
  \section*{\underline{Span, LI, Basis}}
    % 2.20
    \paragraph{Linear combination} A linear combination of vectors $v_i$ is
      \[ a^1v_1+a^2v_2+\cdots+a^nv_n = \sum a^i v_i \]
      Note the use of superscripts instead of subscripts, for the scalars.
    % 2.21
    \paragraph{Span} The span of a list of vectors is the set of all linear combinations of the vectors.
    % 2.25
    \paragraph{Linearly independent} A list of vectors $v_i$ is linearly independent if
      \[ \sum a^iv_i = 0 \quad \Rightarrow \quad \forall i \; (a^i = 0) \]
    % 2.27, 2.28
    \paragraph{Basis} A basis is a span that is linearly independent.
      \ull {
        \item Every finite-dimensional vector space has a basis.
        \item Every vector can be expressed uniquely as a linear combination of the vectors in the basis.
      }
    % 2.31, 2.34
    \paragraph{Decompose into direct sum} Let $U$ be a subspace of a finite-dimensional vector space $W$.
      \\\\
      Then there exists $V$, a subspace of $W$, such that \( W = U \oplus V \), and \( \dim(W) = \dim(U \oplus V) = \dim(U) + \dim(V) \).
    \subsection*{Basis as a mapping}
      A basis can be thought of as a \underline{mapping} $\phi: \mathcal{F}^n \rightarrow V$, i.e. it turns a list of numbers (components) into a vector associated with the basis. This mapping is a vector space isomorphism.
      \\\\
      Using the definition of $\phi$ above, then \( z_i = \phi(e_i) \), where $e_i$ are the canonical basis vectors, forms a basis for $V$. Thus, a basis is just a specific example of the infinitely many vector space isomorphism between $\mathcal{F}^n$ and $V$.
  \section*{\underline{LaTeX stuff}}
    \begin{center}
      \begin{tabular}{ |C|c| }
        \hline
          \begin{pmatrix}
            1 & 0 \\
            0 & 1
          \end{pmatrix}
          &
        \begin{lstlisting}
\begin{pmatrix}
  1 & 0 \\
  0 & 1
\end{pmatrix}
        \end{lstlisting}
        \\ \hline
          \begin{vmatrix}
            1 & 0 \\
            0 & 1
          \end{vmatrix}
          &
          vmatrix
        \\ \hline
          \mathcal{F} \quad \zeta \quad \hat{i} \quad A^T
          &
          \begin{lstlisting}
\mathcal{F} \zeta
\hat{i} A^T
          \end{lstlisting}
        \\ \hline
      \end{tabular}
    \end{center}
\end{multicols*}
\end{document}
