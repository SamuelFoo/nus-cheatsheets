\documentclass[a4paper]{article}
\usepackage[a4paper,
            top=0.4in,
            bottom=0.6in,
            left=0.4in,
            right=0.4in,
            landscape]{geometry}

% default stuff
\usepackage{amsmath}
\usepackage{amssymb}
\usepackage{enumitem}

% multicolumn package
\usepackage{multicol}

\usepackage{multirow}
\usepackage{makecell}
\renewcommand\theadfont{\bfseries}

\newcolumntype{M}[1]{>{\centering\arraybackslash} m{#1}} % centered m

\newcommand{\abs}[1]{\left\lvert#1\right\rvert}

\newcommand{\ol}[1]{\begin{enumerate}#1\end{enumerate}}
\newcommand{\oll}[1]{\begin{enumerate}[leftmargin=*]#1\end{enumerate}}
\newcommand{\ul}[1]{\begin{itemize}#1\end{itemize}}
\newcommand{\ull}[1]{\begin{itemize}[leftmargin=*]#1\end{itemize}} % no margin


\usepackage[dvipsnames,table]{xcolor}
\newcommand{\red}[1]{\textcolor{red}{#1}}
\newcommand{\vv}{\vec{v}}

\begin{document}
\part*{\centering \underline{PC1201 Mechanics}}
\begin{multicols*}{4}
  \small
  \section*{\underline{Miscellaneous}}
    \subsection*{Math}
      \paragraph{Standards}
        \ul {
          \item Use of SI units is encouraged (m, s, kg, A)
          \item Precision of answer should be the same as the \textbf{least} significant value provided in question
        }
      \paragraph{Angles} An angle is conventionally defined by going counter-clockwise from the positive $x$-axis.
      \paragraph{Vectors} Consider a vector $\vv$, making an angle $\theta$ with respect to the positive $x$-axis.
        \ul {
          \item Magnitude: $\abs{\vv}$ or $\displaystyle \sqrt{(\vv_x)^2 + (\vv_y)^2}$
          \item Direction $\theta$ or $\displaystyle \tan\theta = \dfrac{\vv_y}{\vv_x}$ (take care of sign of tangent appropriately)
          \item Usually 2 components in PC1201
          \item Addition: attach tails, sum up each component
          \item Subtraction: adding $-1$ times
          \item Scalar multiplication: scales vector
        }
      \paragraph{Length of vector components} Consider a vector $\vv$, making an angle $\theta$ with respect to the positive $x$-axis.
        \ul {
          \item Hypothenuse: $\abs{\vv}$
          \item Adjacent: $\abs{\vv} \cos\theta$
          \item Opposite: $\abs{\vv} \sin\theta$
        }
    \subsection*{Problem solving}
      \paragraph{Tilt axes} Do not always need to choose ``standard" cartesian axis with $x$ direction parallel to ground.
      \paragraph{Decompose into axes} Sometimes, it is possible to decompose an event into perpendicular components and we should do so.
  \section*{\underline{Kinematics}}
    \paragraph{Ticker tape} Dots are drawn every fixed interval.
      \ul {
        \item Larger gap $\Rightarrow$ faster speed
        \item Increasing gap size $\Rightarrow$ positive acceleration
      }
    \paragraph{Scalar vs vector}
      \begin{center}
        \begin{tabular}{ |c|c| }
          \hline
          \thead{Scalar} & \thead{Vector} \\ \hline
          Distance & Displacement \\ \hline
          Speed & Velocity \\ \hline
          - & Acceleration \\ \hline
        \end{tabular}
      \end{center}
      \paragraph{Graphs} We consider position-time ($x$-$t$), velocity-time ($v$-$t$), acceleration-time ($a$-$t$). Relations (due to calculus)
        \ul {
          \item Gradient of $x$-$t$ is velocity
          \item Gradient of $v$-$t$ is acceleration
          \item Area under $v$-$t$ is displacement
        }
      \paragraph{SUVAT}
        \[
          \begin{aligned}
            s &= ut + \frac{1}{2}at^2 \\
            v &= u + at \\
            v^2 &= u^2 + 2as \\
            s &= \frac{1}{2} (u+v) t
          \end{aligned}
        \]
        Note: we generally deal with constant acceleration, so if there are sudden changes in acceleration, we deal with the motion separately.
    \subsection*{Projectile motion}
      \paragraph{Trajectory} Parabola, only depending on the initial velocity $\vec{u}$.
      \[ y = x \tan \theta - \frac{gx^2}{2u^2 \cos^2\theta} \]
        \ul {
          \item Velocity is tangent to trajectory.
          \item $\vv_y = 0$ at top of trajectory.
          \item $\vv_x$ is constant throughout trajectory.
        }
      \paragraph{Height and range}
        \[
          \begin{aligned}
            H &= \frac{u^2 \sin^2\theta}{2g} \\
            R &= \frac{u^2 \sin 2\theta}{g}
          \end{aligned}
        \]
        Hence, $\theta = 90\deg$ gives max height, while $\theta = 45\deg$ gives max range.
    \subsection*{Circular motion}
      \paragraph{Properties}
        \ul {
          \item Velocity is not constant, as direction is always changing
          \item Thus an acceleration exists:
            \[ a_c = \frac{\abs{\vv}^2}{r} \]
            This is known as centripetal acceleration, and its direction is perpendicular to $\vv$.
        }
      \paragraph{Non-uniform}
        \ol {
          \item Implies speed is changing
          \item Implies net acceleration is not perpendicular to velocity
          \item $\abs{\vv}$ is not constant
        }
    \subsection*{Relative motion}
      Velocity is technically a relative quantity. We use the following notation:
      \[ \vv_{\text{object} \vert \text{reference frame}} \]
      We can compare relative velocities:
      \[ \vv_{A\vert B} = \vv_{A\vert O} - \vv_{B\vert O} = \vv_{A\vert O} + \vv_{O\vert B} \]
    \subsection*{Problem solving}
      \paragraph{List information} Before trying a question, list all information.
        \ul {
          \item Number of unknowns determines number of equations to use
          \item Draw diagrams for understanding
          \item Name unknowns for easy reference
          \item Not all information is numerical (Starts from rest $\Rightarrow \vec{u} = 0$)
        }
  \section*{\underline{Dynamics}}
    Newton can be expressed in base units:
    \[ \mathrm{N} = \mathrm{kg}\ \mathrm{m}\ \mathrm{s}^{-2} \]
    \subsection*{Forces}
      \paragraph{Gravity} \( W = mg \)
      \paragraph{Tension} A contact force that is same throughout the rope. The tension at which the rope breaks is called breaking tension.
      \paragraph{Normal} A contact force that acts perpendicular to the surface.
      \paragraph{Friction} A contact force that acts opposite to the motion, resisting
        \ul {
          \item Motion (Kinetic, $f_k$)
          \item Tendency to move (Static, $f_s$)
        }
        The general equation is:
        \[ f = \mu N \]
        where $\mu$ is the relevant friction coefficient.
        \ul {
          \item Static friction is equal to the applied force until a certain $f_{s_\text{max}} = \mu_s N$.
          \item Once the applied force exceeds this, the object starts moving, and kinetic friction is now present, instead of static friction, and now $f_k = \mu_k N$, where $\mu_k \leq \mu_s$.
        }
        \paragraph{Elastic} Known as a restoring force, acting in the opposite direction of $\Delta \vec{x}$.
        \[ \vec{F} = -k \Delta \vec{x} \]
    \subsection*{Circular motion}
      The net force that causes circular motion is known as centripetal force.
    \subsection*{Orbital motion}
      \ul {
        \item If we launch a projectile with a high enough speed, it can escape gravity.
        \item If we get the speed right, the object will be so fast that as it falls towards the earth, the earth starts to curve away.
        \item If it is too slow, the object will eventually return to Earth.
        \item If it is too fast, the object escapes Earth's gravity.
        \item An object in such a circular orbit is in constant free fall, only being acted on by gravity. Hence,
          \begin{align*}
            mg &= \frac{mv^2}{r} \\
            v &= \sqrt{rg}
          \end{align*}
      }
    \subsection*{Apparent weight}
      \paragraph{Normal force} Our apparent weight changes if $N \neq mg$.
        \ul {
          \item If $N < mg$, we feel lighter (low apparent weight)
          \item If $N > mg$, we feel heavier (high apparent weight)
        }
        We can experience this in elevators, vertical circular motion, orbitting satellites, etc.
    \subsection*{Problem solving}
      \ul {
        \item Draw sketch of entire system
        \item Isolate SINGLE body to draw FBD, with forces acting ON the body
        \item Action-reaction pairs should NOT appear in same FBD
        \item Choose axis that aligns with net acceleration
        \item If object is accelerating, net force is nonzero
      }
  \section*{\underline{Work, Energy, Power}}
    \subsection*{Energy}
      \ul {
        \item Examples: Kinetic, Potential (Gravitational, Elastic, Electric), Thermal, Chemical
        \item Unit: Joules (J)
        \item Is a scalar
        \item Energy transformation - changes within the system
        \item Energy transfer - exchanges with external bodies outside system
      }
    \subsection*{Work}
      \ul {
        \item Work is transfer of energy - done on a system by external force
        \item Is a scalar, but positive work should increase the total energy of the system
        \item No movement = no work done
        \item The equation is given by
          \[ W = F_{\vert\vert}s = Fs \cos\theta \]
          where $F_{\vert\vert}$ is the component of the force parallel to displacement.
        \item Work is positive if the force is acting in the same direction as displacement
        \item Work done is area under graph of $F-x$
      }
    \subsection*{Types of energy}
      \paragraph{Kinetic} $W = \frac{1}{2}mv^2$
      \paragraph{Potential energy}
        \ul {
          \item Stored energy
          \item Allowed by conservative forces (where work done is dependent on position, and independent of path)
          \item Examples of conservative forces: Gravitational, elastic, electric (and magnetic) forces
          \item Derive formula by using
            \[ \abs{W} = \abs{F \Delta s} \]
          \item When conservative forces do positive work, potential energy of object decreases
        }
      \paragraph{Gravitational potential} $\Delta U_g = mg \Delta h$
      \paragraph{Elastic potential} $U_s = \frac{1}{2} k (\Delta x)^2$
    \subsection*{Conservation of Energy}
      \ul {
        \item In a closed system, total energy is constant.
        \item In a open system, we consider only mechanical energy, ignoring heat.
      }
    \subsection*{Power}
      \ul {
        \item Rate of energy transfer
        \[ P = \frac{\Delta E}{t} = \frac{W}{t} \]
        \item For kinematics, if object is moving at constant speed,
        \[ P = \frac{Fs}{t} = Fv \]
        \item Unit: Watt (W)
      }
    \subsection*{Problem solving}
      \ul {
        \item Focus is on before and after, ignoring what happens in between
        \item Be aware if system is open or close
      }
  \section*{\underline{Momentum}}
    \ul {
      \item Momentum is a vector quantity that quantifies motion:
        \[ \vec{p} = m \vv \]
      \item Unit: kg m s$^{-1}$
    }
    \subsection*{Conservation of Momentum}
      \ul {
        \item In a closed system, momentum of the \textbf{total system} is conserved
        \item But momentum of individual objects in the system may change
        \item Can decompose into perpendicular axes
      }
    \subsection*{Types of collision}
      \paragraph{Elastic} KE is conserved, objects separate.
      \\\\
      Consider two objects with mass $m_i$, initial velocity $u_i$ and final velocity $v_i$.
        \begin{align*}
          m_1 \vec{u_1} + m_2 \vec{u_2} &= m_1 \vec{v_1} + m_2 \vec{v_2} \\
          \frac{1}{2} m_1 u_1^2 + \frac{1}{2} m_2 u_2^2 &= \frac{1}{2} m_1 v_1^2 + \frac{1}{2} m_2 v_2^2 \\
        \end{align*}
        By algebra,
        \[ \vec{v_2} - \vec{v_1} = -(\vec{u_2} - \vec{u_1}) \]
        Thus, the relative velocity after collision is same in magnitude, and opposite in direction.
      \paragraph{Completely inelastic} KE is not conserved, objects stick
      \paragraph{Inelastic} KE is not conserved, objects separate (occurs when one body splits into two)
    \subsection*{Impulse}
      Momentum is not conserved in open systems, and the change in momentum is called impulse.
      \[ \vec{J} = \vec{F}_\text{avg} \Delta t \]
      It is also the area under the $F$-$t$ graph.
\end{multicols*}
\end{document}
