\documentclass[a4paper]{article}
\usepackage[a4paper,
            top=0.4in,
            bottom=0.6in,
            left=0.4in,
            right=0.4in,
            landscape]{geometry}

% default stuff
\usepackage{amsmath}
\usepackage{amssymb}
\usepackage{enumitem}

% multicolumn package
\usepackage{multicol}

\usepackage{multirow}
\usepackage{makecell}
\renewcommand\theadfont{\bfseries}

\newcolumntype{M}[1]{>{\centering\arraybackslash} m{#1}} % centered m

\newcommand{\abs}[1]{\left\lvert#1\right\rvert}

\newcommand{\ol}[1]{\begin{enumerate}#1\end{enumerate}}
\newcommand{\oll}[1]{\begin{enumerate}[leftmargin=*]#1\end{enumerate}}
\newcommand{\ul}[1]{\begin{itemize}#1\end{itemize}}
\newcommand{\ull}[1]{\begin{itemize}[leftmargin=*]#1\end{itemize}} % no margin

% coloring code
\usepackage{color}
\definecolor{dkgreen}{rgb}{0,0.6,0}
\definecolor{gray}{rgb}{0.5,0.5,0.5}
\definecolor{mauve}{rgb}{0.58,0,0.82}

\definecolor{lg}{rgb}{0.9,0.9,0.9}

% code environment
\usepackage{listings}
\lstset{
  %frame=tb, % adds top and bottom border
  aboveskip=1mm,
  belowskip=1mm,
  showstringspaces=false,
  columns=flexible,
  basicstyle={\small\ttfamily},
  numberstyle=\color{gray},
  keywordstyle=\color{blue}\textbf,
  commentstyle=\color{dkgreen},
  stringstyle=\color{mauve},
  breaklines=true,
  breakatwhitespace=true,
  tabsize=4
}
\newcommand{\ic}[1]{\lstinline{#1}}

% drawing a stack diagram
\usepackage{drawstack}
\usetikzlibrary{arrows.meta}

\newcommand{\ptr}[1]{${}^{*}$#1}
\newcommand{\addr}[1]{\&#1}
\newcommand{\addrb}[1]{(\&#1)}

% special table column for code
\usepackage{collcell}
\newcommand*{\tablecode}[1]{\lstinline{#1}}% Do anything you like with `#1`
\newcolumntype{T}{>{\collectcell\tablecode}c<{\endcollectcell}}


% mips
\usepackage{mips}

% k-map
\usepackage{karnaugh-map}

\begin{document}
\begin{multicols*}{3}
  \small
  \subsection*{C language}
    \lstset{language=C}
    \paragraph{Misc}
      \ul {
        \item \ic{int x = -10 \% 4; //gives -2}
        \item $\sim$ is binary NOT, ! is boolean NOT
        \item When passing an array to a function, it turns into a pointer to the first element of the array.
      }
    \paragraph{Format specifiers}
      \begin{center}
        \begin{tabular}{ |T|T|c| }
          \hline
          \multicolumn{1}{|c|}{\thead{Spec.}} & \multicolumn{1}{c|}{\thead{Variable Type}} & \thead{Function Use} \\
          \hline
          \%c & char & printf / scanf \\
          \hline
          \%d & int & printf / scanf \\
          \hline
          \%f & float / double & printf \\
          \hline
          \%f & float & scanf \\
          \hline
          \%lf & double & scanf \\
          \hline
          \%e & float / double & printf (scientific notation) \\
          \hline
        \end{tabular}
      \end{center}
      Examples of format specifiers used in \ic{printf()}:
      \ul {
        \item \ic{\%5d}: to display an integer in a width of 5, right justified
        \item \ic{\%8.3f}: to display a real number (float or double) in a width of 8, with 3 decimal places, right justified
      }
    \paragraph{Escape sequences}
      \begin{center}
        \begin{tabular}{ |c|c|M{4.2cm}| }
          \hline
          \thead{Seq.} & \thead{Meaning} & \thead{Result} \\
          \hline
          \ic{\\n} & New line & Subsequent output will appear on the next line \\
          \hline
          \ic{\\t} & Horizontal tab & Move to the next tab position on the current line \\
          \hline
          \ic{\\"} & Double quote & Display a double quote \ic{"} \\
          \hline
          \ic{\%\%} & Percent & Display a percent character \ic{\%} \\
          \hline
        \end{tabular}
      \end{center}
    \paragraph{Mixed-Type Arithmetic}.
      \begin{lstlisting}
int m = 10/4;     // m = 2
float p = 10/4;   // p = 2.0
int n = 10/4.0;   // n = 2
float q = 10/4.0; // q = 2.5
int r = -10/4.0;  // r = -2
      \end{lstlisting}
    \paragraph{Type Casting}.
      \begin{lstlisting}
int aa = 6; float ff = 15.8;
float pp = (float) aa / 4;   // pp = 1.5
int nn = (int) ff / aa;      // nn = 2
float qq = (float) (aa / 4); // qq = 1.0
      \end{lstlisting}
    \paragraph{Order of operations}
      The operators are ranked according to precedence, with the top row having the highest precedence.
      \\\\
      Operators in the same row have the same precedence, and they are evaluated in the given associativity direction.
      \begin{center}
        \begin{tabular}{ |c|T|c| }
          \hline
          \textbf{Type} & \multicolumn{1}{c|}{\textbf{Operator}} & \textbf{Assoc} \\
          \hline
          Primary expr & () [] . -> expr++ expr-- & L2R \\
          \hline
          \multirow{2}{*}{Unary} & * \& + - ! ~ ++expr & \multirow{2}{*}{R2L} \\
                                 & --expr sizeof (typecast) & \\
          \hline
          \multirow{10}{*}{Binary} & * / \% & \multirow{10}{*}{L2R} \\
          \cline{2-2}
                                  & + - & \\
          \cline{2-2}
                                  & << >> & \\
          \cline{2-2}
                                  & < > <= >= & \\
          \cline{2-2}
                                  & == != & \\
          \cline{2-2}
                                  & \& & \\
          \cline{2-2}
                                  & ^ & \\
          \cline{2-2}
                                  & | & \\
          \cline{2-2}
                                  & \&\& & \\
          \cline{2-2}
                                  & || & \\
          \hline
          Ternary & ?: & R2L \\
          \hline
          Assignment & = += -= *= /= \%= <<= >>= & R2L \\
          \hline
        \end{tabular}
      \end{center}
  \subsection*{Number Systems}
    \paragraph{Sign extension} Used in 2's complement representation. For example, to represent a 4-bit 2's complement using 8 bits,
      \begin{equation*}
        0110_{2s} \rightarrow 00000110_{2s}
      \end{equation*}
      Sign extension is value-preserving. For example,
      \begin{align*}
        00000110_{2s} &= 0110_{2s} \\
        11111010_{2s} &= 1010_{2s}
      \end{align*}
    \paragraph{Rounding in fixed-point} 2's complement fixed point representation, total 8 bits, whole number part 6 bits, fractional part 2 bits, want to represent $26.875_{10}$. Without restriction,
      \begin{equation*}
        26.875_{10} = 11011.111_2
      \end{equation*}
      Since we have 6 bits for whole number part, it is $011011$. But we only have 2 bits in the fractional part to represent $.111$. Since the bit immediately after the 2nd bit is a 1, we round up, so we get $1.00$.
  \subsection*{MIPS}
    \lstset{language=[mips]Assembler}
    MIPS is big-endian, so it stores bytes in its natural order.
    \paragraph{Get certain bits}
      \ic{andi} can be used to obtain specific bits from a variable.
      \ul {
        \item Place 0 into positions to be ignored, $X \cdot 0 = 0$
        \item Place 1 into interested positions, $Y \cdot 1 = Y$
      }
      \ic{andi $t0, $t0, mask}
    \paragraph{Force certain bits to 1}
      \ic{ori} can be used to force certain bits to be 1
      \ul {
        \item Place 0 into positions to be left as-is, $X + 0 = X$
        \item Place 1 into interested positions, $Y + 1 = 1$
      }
      \ic{ori $t0, $t0, mask}
    \paragraph{Perform NOT}
      \ul {
        \item \ic{nor $t0, $t0, \$zero}
        \item \ic{xor $t0, $t0, mask}, where mask is all 1s
      }
    \paragraph{Load 32-bit constant into register}
      MIPS only allows up to 16 bits in immediate, so we cannot directly load a 32-bit constant into register.
      \begin{lstlisting}
lui $t0, upper_16_bits
ori $t0, $t0, lower_16_bits
      \end{lstlisting}
    \paragraph{Load value from address} .
      \begin{lstlisting}
la $t0, addr
lw $s0, 0($t0)
      \end{lstlisting}
    \paragraph{Immediate parameter}
      \ul {
        \item \ic{addi} accepts immediate in the range $[-2^{15}, 2^{15}-1]$
        \item \ic{sll}, \ic{srl} accepts shift in the range $[0, 2^5-1]$
        \item \ic{andi}, \ic{ori} treat immediate as 16-bit pattern
      }
  \subsection*{Boolean Algebra}
    \paragraph{Gray Code}
      \ul {
        \item Single bit change from one code value to the next
        \item Standard gray code for $n$ bits can be obtained by listing the gray codes for $n-1$ bits in natural order, followed by the gray codes for $n-1$ bits in reverse order. Prepend 0 to the first half of the list, and prepend 1 to the second half of the list.
        \item Binary to Standard Gray: \ic{x XOR (x >> 1)}
        \item Standard Gray to Binary: \ic{x XOR (x >> 1) XOR (x >> 2) ... XOR 0}
      }
    \columnbreak
    \paragraph{Karnaugh Maps}
      \begin{center}
        \begin{karnaugh-map}[4][4][1][$CD$][$AB$]
          \manualterms{m0,m1,m2,m3,m4,m5,m6,m7,m8,m9,m10,m11,m12,m13,m14,m15}
        \end{karnaugh-map}
      \end{center}
      3-var K-map is just the top half. 5-var K-map is two squares, with the second square offset by 16.
    \paragraph{Truth tables}
      \begin{center}
        \begin{tabular}{ |c|c|c|c|c|c| }
          \hline
          A & B & XOR & NOR & NAND & XNOR \\
          \hline
          0 & 0 & 0 & 1 & 1 & 1 \\
          0 & 1 & 1 & 0 & 1 & 0 \\
          1 & 0 & 1 & 0 & 1 & 0 \\
          1 & 1 & 0 & 0 & 0 & 1 \\
          \hline
        \end{tabular}
      \end{center}
      XOR is commutative, associative. A XOR A = 0 and A XOR 0 = A.
  \subsection*{Combinatorial/Sequential}
    \begin{center}
      \begin{tabular}{ |cc|c|c| }
        \hline
        S & R & Q(t+1) & Behaviour \\
        \hline
        0 & 0 & Q(t) & No change \\
        0 & 1 & 0 & Reset \\
        1 & 0 & 1 & Set \\
        1 & 1 & indeterminate & Invalid condition \\
        \hline
      \end{tabular}
      \begin{equation*}
        Q(t+1) = S + R' \cdot Q
      \end{equation*}
      \begin{tabular}{ |c|c|c| }
        \hline
        D & Q(t+1) & Behaviour \\
        \hline
        0 & 0 & Reset \\
        1 & 1 & Set \\
        X & Q(t) & No change \\
        \hline
      \end{tabular}
      \begin{equation*}
        Q(t+1) = D
      \end{equation*}
      \begin{tabular}{ |cc|c|c| }
        \hline
        J & K & Q(t+1) & Behaviour \\
        \hline
        0 & 0 & Q(t) & No change \\
        0 & 1 & 0 & Reset \\
        1 & 0 & 1 & Set \\
        1 & 1 & Q(t)' & Toggle \\
        \hline
      \end{tabular}
      \begin{equation*}
        Q(t+1) = J \cdot Q' + K' \cdot Q
      \end{equation*}
      \begin{tabular}{ |c|c|c| }
        \hline
        T & Q(t+1) & Behaviour \\
        \hline
        0 & Q(t) & No change \\
        1 & Q(t)' & Toggle \\
        \hline
      \end{tabular}
      \begin{equation*}
        Q(t+1) = T \cdot Q' + T' \cdot Q
      \end{equation*}
    \end{center}
  \subsection*{Pipelining}
    Jump is NOT a control hazard because it does not dictate which instruction to execute, rather it just causes a delay.
    \subsubsection*{Time for each stage}
      \begin{center}
        \begin{tabular}{ |c|c|c|c|c|c|c| }
          \hline
          \thead{Inst.} & \thead{IF} & \thead{ID} & \thead{EX} & \thead{MEM} & \thead{WB} & \thead{Total} \\
          \hline
          ALU & 2 & 1 & 2 & - & 1 & 6 \\
          \hline
          lw & 2 & 1 & 2 & 2 & 1 & 8 \\
          \hline
          sw & 2 & 1 & 2 & 2 & - & 7 \\
          \hline
          beq & 2 & 1 & 2 & - & - & 5 \\
          \hline
        \end{tabular}
      \end{center}
    \subsubsection*{Reducing stall for branching}
      \paragraph{Early Branch Resolution}
        \ul {
          \item Move branch decision calculation from MEM to ID stage, (usually) stall 1 cycle instead of 3.
          \item The early branch calculation might result in a data hazard, so it might stall for 2 cycles.
          \item In an even worse case, if the instruction before the branch is \ic{lw}, we need to stall for 3 cycles even with early branching, so there is no improvement.
        }
      \paragraph{Branch Prediction}
        \ul {
          \item Guess outcome, and if it is right, then there is no pipeline stall. Otherwise flush the wrong instructions from the pipeline.
          \item The delay for wrong guess depends on whether there is early branch resolution or not.
        }
      \paragraph{Delayed Branch}
        \ul {
          \item Find non-control dependent instructions, and place them after the branch. Moving these instructions should not change program behaviour.
          \item If there are no such instructions, then add no-ops.
        }
  \subsection*{Cache}
    \subsubsection*{Direct Mapped Cache}
      \begin{center}
        \begin{tabular}{ |*{3}{M{2cm}|} }
          \multicolumn{2}{|c|}{
            $\xleftarrow{\hspace{0.5cm}}$
            Block number
            $\xrightarrow{\hspace{0.5cm}}$
          } \\
          \hline
          Tag & Index & Offset \\
          \hline
        \end{tabular}
      \end{center}
      Given cache block size = $2^N$ bytes, and number of cache blocks = $2^M$. Then, offset is $N$ bits, index is $M$ bits, and tag is $32 - (N+M)$ bits.
      \paragraph{Overhead} Valid flag (1 bit) + Tag length, for each block
    \subsubsection*{Set Associative Cache}
      \begin{center}
        \begin{tabular}{ |*{3}{M{2cm}|} }
          \multicolumn{2}{|c|}{
            $\xleftarrow{\hspace{0.5cm}}$
            Block number
            $\xrightarrow{\hspace{0.5cm}}$
          } \\
          \hline
          Tag & Index & Offset \\
          \hline
        \end{tabular}
      \end{center}
      Given cache block size = $2^N$ bytes, and number of cache sets = $2^M$. Then, offset is $N$ bits, set index is $M$ bits, and tag is $32 - (N+M)$ bits.
    \subsubsection*{Fully Associative Cache}
      \begin{center}
        \begin{tabular}{ |M{4cm}|M{2cm}| }
          \multicolumn{1}{|c|}{
            $\xleftarrow{\hspace{0.5cm}}$
            Block number
            $\xrightarrow{\hspace{0.5cm}}$
          } \\
          \hline
          Tag & Offset \\
          \hline
        \end{tabular}
      \end{center}
      Given cache block size = $2^N$ bytes, and number of cache blocks = $2^M$. Then, offset is $N$ bits, and tag is $32 - N$ bits.
      \paragraph{Overhead (Set Assoc. and Fully Assoc.)} Valid flag (1 bit) + Tag length, for each block in set
      \\\\
      Note that in a fully associative cache, all blocks are searched simultaneously.
  \end{multicols*}
\end{document}
